\chapter{Summary}
\label{cha:Summary}

This thesis aimed to evaluate the current state of knowledge and compare available methods for data augmentation, as well as their impact on the performance of image classifiers. The scope of the work included assessing the effectiveness of traditional, advanced, and GAN-based methods for two different state-of-the-art architectures, based on experiments conducted on several datasets. The thesis successfully achieved this goal by conducting an in-depth analysis of the literature and experiments on data augmentation and presenting the available methods and results.

In the experiments, three different scenarios were tested. The first dataset was a standard set with over 100 categories, posing a significant challenge for the networks. The second dataset was used to evaluate the efficiency of augmentation methods under limited data conditions. The third dataset consisted of audio files translated into mel spectrograms, enabling image-based analysis and allowing for the testing of additional augmentation techniques in a domain-specific context.

Two state-of-the-art networks, \textit{ResNet50} and \textit{EfficientNetB0}, were trained on each dataset, and high-quality models, which required extensive fine-tuning, were obtained. This rigorous approach ensured that the networks were highly optimized, demonstrating the effectiveness of the augmentation methods and the robustness of these architectures.

Various evaluation methods were used in each experiment, including numerical metrics such as AUC, accuracy, and F1 score. Additionally, learning curves, confusion matrices, and saliency maps were employed. The use of saliency maps was particularly innovative, providing visual insights into the model's focus areas and enhancing the understanding of how augmentations impacted model performance. The metrics indicated that proper application of augmentation techniques leads to an increase in model performance. Furthermore, even when the metrics remain unchanged, the model benefits from exposure to a wider range of scenarios, improving its capabilities to handle new data from different distributions.

In terms of GAN-based augmentation, due to the time and resource consumption, experiments conducted by other researchers were briefly described, and common conclusions were drawn. While traditional augmentation is preferred in most cases due to its simplicity and reliable results, GAN-based augmentation is beneficial when dealing with highly imbalanced datasets, small amounts of data, or medical data where information from other datasets can be transferred to our case. GAN-based augmentation is also advantageous when a certain type of data is difficult to obtain.

Future work could involve conducting more experiments focused on GANs or exploring other types of tasks beyond classification. This could provide deeper insights into the versatility and broader applications of data augmentation techniques.